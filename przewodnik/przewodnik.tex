% By zmienic jezyk na angielski/polski, dodaj opcje do klasy english lub polish
\documentclass[polish,12pt]{aghthesis}
\usepackage[utf8]{inputenc}
\usepackage{url}

\author{Piotr Góralczyk, Paweł Mikołajczyk\\ Dominik Rusiecki}

\title{System do analizy i monitorowania\\ portali ogłoszeniowych}

\supervisor{dr inż.\ Jacek Dajda}

\date{2015}

% Szablon przystosowany jest do druku dwustronnego, 
\begin{document}

\maketitle



\section{Cel prac i wizja produktu}
%\section{Project goals and vision}
\label{sec:cel-wizja}
\emph{Charakterystyka problemu, motywacja projektu (w tym przeglad
  istniejących rozwiązań prowadząca do uzasadnienia celu prac), ogólna
  wizja produktu, krótkie studium wykonalności i analiza zagrożeń.}

Problemem, z którym stara się mierzyć system stworzony w ramach niniejszej
pracy inżynierskiej, jest obserwacja wielu źródeł informacji w Sieci w 
poszukiwaniu interesujących użytkownika zagadnień z danej tematyki. Klient 
wyraził pragnienie posiadania narzędzia, umożliwiającego mu monitorowanie 
stron i portali internetowych pod kątem zdefiniowenego przez niego zbioru 
informacji, a ponadto udostępniającego metody analizy tych informacji. 

Istnieją narzędzia odpowiadające na podobne potrzeby - Google Search pozwala 
nam na poszukiwanie informacji w Internecie w odpowiedzi na podane hasło, a
Google Analytics może służyć do analizowania trendów w Sieci.
Nie dają one jednak użytkownikowi możliwości w bardziej specyficzny sposób
definiowania sposobu ekstrakcji danych ze stron. Nie są też w stanie gromadzić
zebranych informacji i w wygodny sposób przeprowadzać na nich analizy. 

W ramach tworzonego projektu powstałby system, dający użytkownikowi swobodę
definiowania rodzaju informacji, jakie będą przezeń poszukiwane, a także 
szczegółowego sposobu ich ekstrakcji ze źródła. Zakładana jest pewna znajomość
 przez użytkownika mechanizmów dopasowania wzorca, jak XPath lub wyrażenia 
regularne. Proces ektrakcji oparty będzie o web-crawling przy użyciu jednego
z rozwiązań open-source lub z wykorzystaniem naszego własnego crawlera. 
Wyekstrahowane informacje przechowywane będą w postaci dedykowanej bazy danych, 
z której będą korzystały narzędzia do analizy. Na wspomnianą analizę ma się 
składać np. wyszukiwanie w zgromadzonych danych słowa kluczowego lub pokazywanie
 trendów czasowych.

Wizja wypracowana w wyniku naszej współpracy z klientem wydawała się do spełnienia.
Część, która mogła nam sprawić dużo problemów, to moduł analizy, wymagający
zarówno złożonych algorytmów po stronie logiki systemu, jak i atrakcyjnego
sposobu prezentacji w formie odpowiedniej technologii frontendowej. Nasze 
doświadczenie na tym polu było niewielkie.

Ponadto, podczas specyfikacji wizji systemu zidentyfikowaliśmy następujące zagrożenia:
\begin{itemize}
\item Brak dostatecznej wiedzy i/lub doświadczenia w takich dziedzinach jak nierelacyjne bazy danych, crawling, technologie front-endowe.
\item Potrzeba łatwego skalowania systemu ze względu na ilość przechowywanych informacji.
\item Konieczność współpracy z różnymi formatami ogłoszeń, które mogą ponadto być zmienne w czasie.
\item Uzależnienie od pracy portali - w przypadku trudności z połączeniem z serwerem, nasz system może przestać działać.
\end{itemize}

\section{Zakres funkcjonalności}
%\section{Functional scope}
\label{sec:zakres-funkcjonalnosci}

\emph{Kontekst użytkowania produktu (aktorzy, współpracujące systemy)
  oraz najważniejsze wymagania funkcjonalne i niefunkcjonalne.}

Aktorem dla naszego systemu jest użytkownik, chcący uzyskać z określonych stron interesujące go informacje.
Jako współpracujące z nim inne systemy należy wyróżnić portale ogłoszeniowe, które użytkownik stargetyzuje
w celu pobrania z nich informacji. Ten fakt narzuca na nas konieczność obsługi danych różnych formatów.

Najważniejsze wymagania funkcjonalne:
\begin{itemize}
\item Użytkownik ma mieć możliwość definiowania danych poprzez określenie ich atrybutów,
sposobu ich ekstrakcji z konkretnego źródła oraz adresu strony, z której zostaną
wyekstrahowane. Estrakcja jest określana za pomocą mechanizmów takich jak XPath lub wyrażenia 
regularne.
\item Użytkownik ma mieć możliwość uruchamiania zadania powiązanego z wyżej
wymienionymi danymi w łatwy sposób.
\item Dla zgromadzonych ogłoszeń, użytkownik ma mieć możliwość przeszukiwania ich po
interesujących go frazach oraz stosować inne, bardziej wyrafinowane metody analizy.
\end{itemize}

Wymagania niefunkcjonalne:
\begin{itemize}
\item Interfejs webowy.
\item Prostota użytkowania.
\item Wydajny crawling.
\item Skalowalność.
\end{itemize}

\section{Wybrane aspekty realizacji}
%\section{Selected realization aspects}
\label{sec:wybrane-aspekty-realizacji}

\emph{Przyjęte założenia, struktura i zasada działania systemu,
  wykorzystane rozwiązania technologiczne wraz z krótkim uzasadnieniem
  ich wyboru.}

\section{Organizacja pracy}
%\section{Work organization}
\label{sec:organizacja-pracy}

\emph{Struktura zespołu (role poszczególnych osób), krótki opis i
  uzasadnienie przyjętej metodyki i/lub kolejności prac, planowane i
  zrealizowane etapy prac ze wskazaniem udziału poszczególnych
  członków zespołu, wykorzystane praktyki i narzędzia w zarządzaniu
  projektem.}

Podczas pracy nie wyodrębnialiśmy konkretnych ról poszczególnych osób. Zamiast tego, raczej staraliśmy się dzielić pracą, niezależnie czy
był to
research, development, testy, wdrażanie systemu czy tworzenie dokumentacji.

W aspekcie procesu wytwarzania oprogramowania przyjęliśmy podejście podobne do procesu ewolucyjnego. W pierwszej fazie tworzenia systemu 
stworzyliśmy prosty prototyp, tak, aby wyklaryfikować niektóre z wymagań klienta, oraz pozyskać wiedzę na temat możliwości oferowanych
przez wybrane przez nas technologie. Następnie, po konsultacjach z klientem, postanowiliśmy rozwijać dalej rzeczony prototyp, dodając 
kolejne funkcjonalności lub modyfikując te już zaimplementowane. 

W pierwszej kolejności skupiliśmy się nad implementacją części systemu, odpowiedzialnej za crawling stron internetowych. Było tak, ponieważ
dopiero po jej sukcesywnej implementacji rozsądne było zastanawiać się nad możliwymi metodami analizy danych. Ze względu jednak na to, że 
ta część systemu pochłonęła dużo naszego czasu, nie wystarczyło nam go, aby w satysfakcjonującym dla klienta stopniu zrealizować moduł analizy.
Stąd zdecydowaliśmy się zrealizować tylko proste wyszukiwanie danych w reakcji na podane przez użytkownika hasło oraz dopracowanie modułu crawlingu,
aby był on łatwiejszy w obsłudze dla nieinformatycznego użytkownika. 

Głównym używanym przez nas narzędziem podczas procesu był system kontroli wersji Git, który umożliwia łatwą pracę wielu osób nad kodem źródłowym.
Ponadto do edycji kodu źródłowego używaliśmy nowoczesnego IDE do Javy firmy JetBrains, IntelliJ IDEA.

\section{Wyniki projektu}
%\section{Project results}

\label{sec:wyniki-projektu}

\emph{Najważniejsze wyniki (co konkretnie udało się uzyskać:
  oprogramowanie, dokumentacja, raporty z testów/wdrożenia, itd.)
  i ocena ich użyteczności (jak zostało to zweryfikowane --- np.\ wnioski
  klienta/użytkownika, zrealizowane testy wydajnościowe, itd.),
  istniejące ograniczenia i propozycje dalszych prac.}

W wyniku prac udało się uzyskać część funkcjonalności
wymaganej początkowo przez klienta. Proces monitorowania odbywa się zgodnie z założeniami,
natomiast, z powodu wspomnianych problemów z czasem, zrealizowane przez nas metody analizy
zgromadzonych danych ograniczone są do wyszukiwania po słowach kluczowych. Stanowi
to więc dobry wektor dla dalszego rozwoju systemu.

Ponadto dla systemu powstała szczegółowa dokumentacja techniczna. Zawiera ona opis
technicznych aspektów systemu, takich jak architektura czy wykorzystane technologie.

Oprócz tego, stworzyliśmy również dokumentację procesową, opisującą przebieg naszych
prac podczas tworzenia systemu, jak i dokumentację użytkownika, mającą ułatwić mu 
zaznajomienie się z systemem i sprawne jego zainstalowanie oraz użytkownie. 

% o ile to mozliwe prosze uzywac odwolan \cite w konkretnych miejscach a nie \nocite

\nocite{artykul2011,ksiazka2011,narzedzie2011,projekt2011}

\bibliography{bibliografia}

\end{document}
