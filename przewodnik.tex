% By zmienic jezyk na angielski/polski, dodaj opcje do klasy english lub polish
\documentclass[polish,12pt]{aghthesis}
\usepackage[utf8]{inputenc}
\usepackage{url}

\author{Piotr Góralczyk, Paweł Mikołajczyk\\ Dominik Rusiecki}

\title{System do analizy i monitorowania\\ portali ogłoszeniowych}

\supervisor{dr inż.\ Jacek Dajda}

\date{2015}

% Szablon przystosowany jest do druku dwustronnego, 
\begin{document}

\maketitle



\section{Cel prac i wizja produktu}
%\section{Project goals and vision}
\label{sec:cel-wizja}
\emph{Charakterystyka problemu, motywacja projektu (w tym przeglad
  istniejących rozwiązań prowadząca do uzasadnienia celu prac), ogólna
  wizja produktu, krótkie studium wykonalności i analiza zagrożeń.}

Problemem, z którym stara się mierzyć system stworzony w ramach niniejszej
pracy inżynierskiej, jest obserwacja wielu źródeł informacji w Sieci w pos
zukiwaniu interesujących użytkownika zagadnień z danej tematyki. Klient wy
raził pragnienie posiadania narzędzia, umożliwiającego mu monitorowanie st
ron i portali internetowych pod kątem zdefiniowenego przez niego zbioru in
formacji, a ponadto udostępniającego metody analizy tych informacji. 

Istnieją narzędzia odpowiadające na podobne potrzeby - Google Search pozwala na
m na poszukiwanie informacji w Internecie w odpowiedzi na podane hasło, a
Google Analytics może służyć do analizowania trendów w Sieci.
Nie dają one jednak użytkownikowi możliwości w bardziej specyficzny sposób
 definiowania sposobu ekstrakcji danych ze stron. Nie są też w stanie groma
dzić zebranych informacji i w wygodny sposób przeprowadzać na nich analizy. 

W ramach tworzonego projektu powstałby system, dający użytkownikowi swobodę
definiowania rodzaju informacji, jakie będą przezeń poszukiwane, a także 
szczegółowego sposobu ich ekstrakcji ze źródła. Zakładana jest pewna znajo
mość przez użytkownika mechanizmów dopasowania wzorca, jak XPath lub wyraże
nia regularne. Proces ektrakcji oparty będzie o web-crawling przy użyciu jednego
z rozwiązań open-source lub z wykorzystaniem naszego własnego crawlera. Wye
kstrahowane informacje przechowywane będą w postaci dedykowanej bazy danych, 
z której będą korzystały narzędzia do analizy. Na wspomnianą analizę ma się 
składać np. wyszukiwanie w zgromadzonych danych słowa kluczowego lub pokazy
wanie trendów czasowych.

Wizja wypracowana w wyniku naszej współpracy z klientem wydawała się do spełnienia.
Część, która mogła nam sprawić dużo problemów, to moduł analizy, wymagający
zarówno złożonych algorytmów po stronie logiki systemu, jak i atrakcyjnego
sposobu prezentacji w formie odpowiedniej technologii frontendowej. Nasze doś
wiadczenie na tym polu było niewielkie. Być może dlatego ta część systemu zosta
ła zaimplementowana tylko po części.

Ponadto, podczas specyfikacji wizji systemu zidentyfikowaliśmy następujące zagrożenia:
\begin{itemize}

\item Ograniczenia dotyczące crawlingu w regulaminach portali ogłoszeniowych.
\item Brak doświadczenia zespołu z nierelacyjnym modelem bazy danych.
\item Konieczność współpracy z różnymi formatami ogłoszeń. Konieczność dostosowywania się do zmian w tych formatach w czasie działania systemu.
\item Uzależnienie od pracy portali może nastąpić brak dostępu do informacji, na który nie mamy wpływu.
\item Stosunkowo szybkie tempo gromadzenia danych wykorzystywanych przez system, co może prowadzić do problemów wydajnościowych.
\item Trudność w implementacji algorytmów wyszukiwania powiązań między ogłoszeniami odpowiadających oczekiwaniom klienta.
\end{itemize}

\section{Zakres funkcjonalności}
%\section{Functional scope}
\label{sec:zakres-funkcjonalnosci}

\emph{Kontekst użytkowania produktu (aktorzy, współpracujące systemy)
  oraz najważniejsze wymagania funkcjonalne i niefunkcjonalne.}

\section{Wybrane aspekty realizacji}
%\section{Selected realization aspects}
\label{sec:wybrane-aspekty-realizacji}

\emph{Przyjęte założenia, struktura i zasada działania systemu,
  wykorzystane rozwiązania technologiczne wraz z krótkim uzasadnieniem
  ich wyboru.}

\section{Organizacja pracy}
%\section{Work organization}
\label{sec:organizacja-pracy}

\emph{Struktura zespołu (role poszczególnych osób), krótki opis i
  uzasadnienie przyjętej metodyki i/lub kolejności prac, planowane i
  zrealizowane etapy prac ze wskazaniem udziału poszczególnych
  członków zespołu, wykorzystane praktyki i narzędzia w zarządzaniu
  projektem.}

\section{Wyniki projektu}
%\section{Project results}

\label{sec:wyniki-projektu}

\emph{Najważniejsze wyniki (co konkretnie udało się uzyskać:
  oprogramowanie, dokumentacja, raporty z testów/wdrożenia, itd.)
  i ocena ich użyteczności (jak zostało to zweryfikowane --- np.\ wnioski
  klienta/użytkownika, zrealizowane testy wydajnościowe, itd.),
  istniejące ograniczenia i propozycje dalszych prac.}

W wyniku prac udało się uzyskać część funkcjonalności
wymaganej początkowo przez klienta. Proces monitorowania odbywa się zgo
dnie z założeniami, natomiast, z powodu problemów technicznych, metody analizy
zgromadzonych danych ograniczone są do wyszukiwania po słowach kluczowych. Stanowi
to dobry wektor dla dalszego rozwoju systemu.

Ponadto dla systemu powstała szczegółowa dokumentacja techniczna. Zawiera ona opis
technicznych aspektów systemu - architektura, wykorzystane technologie.

Oprócz tego, stworzyliśmy również dokumentację procesową, opisującą przebieg naszych
prac podczas tworzenia systemu, jak i dokumentację użytkownika, mającą ułatwić mu 
zaznajomienie się z systemem i sprawne jego zainstalowanie oraz użytkownie. 

% o ile to mozliwe prosze uzywac odwolan \cite w konkretnych miejscach a nie \nocite

\nocite{artykul2011,ksiazka2011,narzedzie2011,projekt2011}

\bibliography{bibliografia}

\end{document}
